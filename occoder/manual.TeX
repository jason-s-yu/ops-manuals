\documentclass[twoside,11pt,letterpaper,abstracton]{scrartcl}
\usepackage{jyu}

\title{Operations Manual}
\subtitle{2019-2020}
\author{\normalsize{\href{https://occoder.org}{OC Coder}}}
\date{\normalsize{Last modified \today \\ \small{Created and maintained by J. Yu}}}

\begin{document}

\thispagestyle{empty}

\maketitle

\section*{Preamble}

Referenced hyperlinks are marked in \color{RoyalBlue} Royal Blue\color{Black}, and can be clicked on to jump to its respective location in the document in any standard document reader. References to a chapter, subsection, or heading are marked by its respective arabic decimal, also highlighted in \color{RoyalBlue} Royal Blue\color{Black}. This document is typeset in Palatino font.

\tableofcontents

\newpage

\section{Processes and Procedures}

Founded in 2013, \emph{Orange County Coder} is a student-run, non-profit organization that strives to promote the understanding of programming and computer science in our everyday lives in order to share our love of computer science to students in Southern California and  surrounding locations.

\subsection{Operational Plans}

Operations in OC Coder are appropriately delegated to its individual departments. More information in \ref{jcc}, \ref{gwc}, \ref{seniorcenter}, and \ref{events}.

Key events include:
\begin{itemize}
    \item Recurring monthly Scratch classes     \ref{jcc-scratch}
    \item Annual Girls Who Code classes         \ref{gwc}
    \item Persistent Senior Center services     \ref{seniorcenter}
    \item The annual Hour of Code in December   \ref{events-hoc}
\end{itemize}

\subsection{Facility Requests}

Per STEAM for All policy, all standard activities, meetings, and gatherings must be held at one Ardent Academy location. Exceptions may be given on a case-by-case basis (see \ref{gwc} (GWC), \ref{seniorcenter} (Senior Center), and \ref{events} (Events)).

Facility request forms can be found here. Please make a copy of the file before editing.

\subsubsection{Ardent Facilities}

Volunteers are expected to maintain the integrity and cleanliness of any checked rooms at Ardent Academy. It is the duty of the Lead Supervisor to ensure that facilities are kept clean and restored to its original condition at the end of any event.

\subsection{Organization Roles and Responsibilities}

\subsubsection{President}

\begin{itemize}
    \item Call and preside over all OC Coder board meetings
    \item (If a co-president is present) Take minutes and timekeep all meetings
    \item Delegate and oversee all functions and operations of OC Coder
    \item Manage and maintain communications between the organization, the student council, and parent volunteers
    \item Oversee all webmaster duties for the organization
\end{itemize}

\subsubsection{Director of Junior Coding Club (3)}

\begin{itemize}
    \item Manage all operations of or relating to the Junior Coding Club department
    \item Recruit, train, and manage all Lead Instructors and volunteers for Scratch classes
    \item Manage all registration relating to Scratch classes, including volunteer and participant tracking
\end{itemize}

\subsubsection{Director of Senior Center (2)}

\begin{itemize}
    \item Manage all operations of and relating to the Senior center department
    \item Manage and facilitate all communications between Orange County Senior centers and OC Coder
    \item Maintain and expand Senior Center operations to more senior centers
    \item Recruit, train, and manage all volunteers participating in Senior Center operations
\end{itemize}

\subsubsection{Director of Girls Who Code (1)}

\begin{itemize}
    \item Manage all operations of and relating to the Girls Who Code department
    \item Manage and facilitate all communications between OC Coder and the Girls Who Code organization
    \item Renew OC Coder membership with the GWC organization annually
    \item Design and oversee Girls Who Code classes for the year, including recruitment and preparation of logistics, instructors, and course materials
\end{itemize}

\subsubsection{Director of Events (2)}

\begin{itemize}
    \item Develop events and topics for any and all educational events
    \item Manage and oversee all miscellaneous OC Coder events
    \item Oversee planning and design of the annual Hour of Code in December, STEAM in the Park, and SFA Open House
\end{itemize}

\newpage

\section{Communications and Repository}

Maintaining communication before, during, and after operations is integral to any organization's success. \emph{OC Coder} primarily uses Slack for day-to-day intraorganization communications. Email is the primary form of communication between the organization and 3rd parties, such as volunteers, other organizations, and other points of contact. 

Organization files are hosted via Google Drive. More information can be found in \ref{com-googledrive}.

\subsection{Communication by Email}

Email templates can be found in Constant Contact.

\subsection{Communication and Workflow Tools}

\subsubsection{Slack}

Slack is a multimedia communication software that facilitates group and direct discussion between members. 

\begin{resource}
    \href{}{To access the organization's Slack workspace, please click here.}
\end{resource}

\subsubsection{Trello}

\href{https://trello.com}{Trello} is a platform that hosts Kanban-style boards to streamline organization workflows. 

At the discretion of OC Coder leadership, Trello may be used to assign and manage tasks, organize files, and share documents. 

\begin{resource}
    \href{}{To access the organization's Trello page, please click here.}
\end{resource}

\subsubsection{Google Drive} \label{com-googledrive}

All OC Coder documents and files can be found in the organization's Google Drive folder. Files are organized as such:

\dirtree{%
    .1 19-20 OC Coder.
        .2 Activities (all miscellaneous activities go here).
        .2 Board Meetings (documents relating to board meetings go here).
            .3 Agendas.
            .3 (meeting minutes go here).
        .2 Events.
            .3 Hour of Code.
            .3 STEAM in the Park.
        .2 Girls Who Code.
        .2 Junior Coding Club.
        .2 Senior Center.
}

\newpage

\section{Chapters and Subsidiaries} \label{schoolchapters}

STEAM for All maintains a series of club-run school chapters. A full list of SFA School Chapters can be found \href{}{here}. OC Coder has a presence in two of these chapters: Sage Hill School and Portola High School.

\newpage

\section{Junior Coding Department} \label{jcc}

\subsection{Instructional Procedures}

\subsubsection{Lead Instructors}

Lead Instructors are expected to set up and teach their designated course. Lead Instructors must:

\begin{itemize}
    \item Prepare and rehearse course material ahead of time
    \item Arrive fifteen minutes prior to the event's start to check-in and begin delegating tasks to volunteers
    \item Reorganize and prepare the classroom for the lessons
\end{itemize}

\subsubsection{General Volunteers}

\begin{itemize}
    \item Review and familarize self with course materials before class
    \item Arrive fifteen minutes prior to the event's start to check-in and help set up the classroom
    \item Remain on standby during the event to ensure that all participants are staying focused and helping students with questions as necessary
\end{itemize}

\subsection{Scratch} \label{jcc-scratch}

Scratch classes occur every month and cover four lessons spread throughout four weeks. The final class is reserved for a hackathon.

\subsubsection{Curriculum}

\begin{resource}
    \href{https://drive.google.com/drive/folders/1tGi0jBzn00RZuYe5zxa5Sv7IjJlqwOqq?usp=sharing}{A repository of all class materials can be found here.} Please refer to below for file references.
\end{resource}

\dirtree{%
    .1 Scratch Lesson Docs.
        .2 Scratch Home Exercises (all homework assignments).
        .2 Scratch Lesson Plans (lead instructor lessons).
        .2 Scratch Parent Survey.
        .2 Scratch Hackathon Rubric.
}

\subsection{JavaScript} \label{jcc-js}

\subsubsection{Curriculum}

\begin{resource}
    \href{https://drive.google.com/drive/folders/12zgeKhSd9gEaBEMIkI69AUVnp-plVbqc?usp=sharing}{A repository of all JS class materials can be found here.}
\end{resource}

\newpage

\section{Events Department} \label{events}

The Events department manages all miscellaneous OC Coder events not directly under the jurisdiction of the other branches. 

\subsection{STEAM in the Park} \label{events-sitp}

STEAM in the Park is the showcase event for all STEAM for All organizations. All organizations contribute activities to a scavenger hunt, where participants go around and participate in as many activities as they can.

\subsection{Hour of Code} \label{events-hoc}

The flagship event of OC Coder, \emph{Hour of Code} is an annual initiative by Code.org. It is held throughout Computer Science Education Week, which is typically the first full week of December.

\begin{resource}
    Materials for the 2018 Hour of Code can be found \href{https://drive.google.com/drive/folders/1VpVb9J50TpbheRf9bMmjayy6Iqel5Pwd?usp=sharing}{here}.
\end{resource}

\subsection{SFA Open House} \label{events-openhouse}

\end{document}